%% LyX 1.1 created this file.  For more info, see http://www.lyx.org/.
%% Do not edit unless you really know what you are doing.
\documentclass[french]{article}
\usepackage[T1]{fontenc}
\usepackage[latin1]{inputenc}
\usepackage{babel}
\usepackage{algorithm}

\makeatletter


%%%%%%%%%%%%%%%%%%%%%%%%%%%%%% LyX specific LaTeX commands.
\providecommand{\LyX}{L\kern-.1667em\lower.25em\hbox{Y}\kern-.125emX\@}

%%%%%%%%%%%%%%%%%%%%%%%%%%%%%% Textclass specific LaTeX commands.
 \usepackage{algolyx}

%%%%%%%%%%%%%%%%%%%%%%%%%%%%%% User specified LaTeX commands.
\algolang{french}

\makeatother

\begin{document}


\title{Algorithmes en fran�ais}


\author{Beno�t Guillon}

\maketitle
\tableofcontents{}

\listofalgorithms{}


\section{Pr�sentation}

Ce court document est un exemple et un test du style Algorithm utilis� en mode
fran�ais (french).


\section{D�finitions par d�faut}

En mode fran�ais, la correspondance entre l'expression logique attendue, ce
qu'il faut �crire, et la sortie est donn�e par le tableau suivant.

\vspace{0.3cm}
{\centering \begin{tabular}{|c|c|c|}
\hline 
Expression logique&
Expression dans le document&
Sortie produite\\
\hline 
\hline 
si&
\textbf{si}&
\textbf{si} ... \textbf{alors}\\
\hline 
sinon&
\textbf{sinon}&
\textbf{sinon}\\
\hline 
sinon si&
\textbf{sinonsi}&
\textbf{sinon si} ... \textbf{alors}\\
\hline 
fin si&
\textbf{finsi}&
\textbf{fin si}\\
\hline 
tant que&
\textbf{tantque}&
\textbf{tant que} ... \textbf{faire}\\
\hline 
fin tant que&
\textbf{fintantque}&
\textbf{fin tant que}\\
\hline 
pour&
\textbf{pour}&
\textbf{pour} ... \textbf{faire}\\
\hline 
fin pour&
\textbf{finpour}&
\textbf{fin pour}\\
\hline 
r�p�ter&
\textbf{r�p�ter}&
\textbf{r�p�ter}\\
\hline 
jusqu'�&
\textbf{jusqu'�}&
\textbf{jusqu'�}\\
\hline 
pour tout&
\textbf{pourtout}&
\textbf{pour tout} ... \textbf{faire}\\
\hline 
boucle infinie&
\textbf{boucle}&
\textbf{boucle}\\
\hline 
fin boucle&
\textbf{finboucle}&
\textbf{fin boucle}\\
\hline 
Contexte&
\textbf{Contexte}&
\textbf{Contexte~:}\\
\hline 
V�rifier&
\textbf{V�rifier}&
\textbf{V�rifier~:}\\
\hline 
\end{tabular}\par}
\vspace{0.3cm}


\section{Exemple}

L'algorithme \ref{alg:exemple} est un exemple d'algorithme en mode fran�ais.

\begin{algorithm}
\begin{algor}
\item [Contexte]condition de d�part
\item [V�rifier]chose � v�rifier en sortie
\item [si]cela est vrai

\begin{algor}
\item [{*}]action 1
\item [pour]10 it�rations

\begin{algor}
\item [{*}]action 2
\end{algor}
\item [finpour]~
\item [r�p�ter]~

\begin{algor}
\item [{*}]action 3
\end{algor}
\item [jusqu'�]ce que ce soit faux
\end{algor}
\item [sinonsi]autre chose

\begin{algor}
\item [tantque]ce test est vrai

\begin{algor}
\item [{*}]it�ration suppl�mentaire
\end{algor}
\item [fintantque]~
\item [pourtout]les �l�ments

\begin{algor}
\item [{*}]action 4
\end{algor}
\item [finpour]~
\end{algor}
\item [sinon]~

\begin{algor}
\item [boucle]~

\begin{algor}
\item [{*}]on reste ici
\end{algor}
\item [finboucle]~
\end{algor}
\item [finsi]~
\end{algor}

\caption{\label{alg:exemple}Algorithme en mode fran�ais}
\end{algorithm}


\end{document}
