%% LyX 1.1 created this file.  For more info, see http://www.lyx.org/.
%% Do not edit unless you really know what you are doing.
\documentclass{article}
\usepackage[T1]{fontenc}
\usepackage[latin1]{inputenc}
\usepackage{algorithm}

\makeatletter


%%%%%%%%%%%%%%%%%%%%%%%%%%%%%% LyX specific LaTeX commands.
\providecommand{\LyX}{L\kern-.1667em\lower.25em\hbox{Y}\kern-.125emX\@}
%% Special footnote code from the package 'stblftnt.sty'
%% Author: Robin Fairbairns -- Last revised Dec 13 1996
\let\SF@@footnote\footnote
\def\footnote{\ifx\protect\@typeset@protect
    \expandafter\SF@@footnote
  \else
    \expandafter\SF@gobble@opt
  \fi
}
\expandafter\def\csname SF@gobble@opt \endcsname{\@ifnextchar[%]
  \SF@gobble@twobracket
  \@gobble
}
\edef\SF@gobble@opt{\noexpand\protect
  \expandafter\noexpand\csname SF@gobble@opt \endcsname}
\def\SF@gobble@twobracket[#1]#2{}

%%%%%%%%%%%%%%%%%%%%%%%%%%%%%% Textclass specific LaTeX commands.
 \usepackage{algolyx}

%%%%%%%%%%%%%%%%%%%%%%%%%%%%%% User specified LaTeX commands.
\algolang{french}
\algoption{noend}
\keycomment{\#<}{>\#}
\makeatother

\begin{document}


\title{Extended use of the Algorithm-Style}

\maketitle
\tableofcontents{}


\section{Introduction}

The purpose of this small file is to test the options and commands that can
be used in the preamble. In the document, we cumulate the three available options:

\begin{itemize}
\item use of noend,
\item specify a language (only french available),
\item redefine the comments delimiters, set to \#< and >\#
\end{itemize}
Moreover, this file checks if some particular features are supported. The tested
features are:

\begin{itemize}
\item use of a very long extra parameter, followed by a comment.
\item use of a very long state item, followed by a comment.
\item use of a very long comment after an endif item.
\item use of a caption at the top of the algorithm float.
\end{itemize}

\section{Examples}


\subsection{Comments, long extra parameter, long state line}

The following algorithm is a small example that should work.

\begin{algor}[1]
\item [si]something is true. The method to decide wether it is true or not is not
very easy. You sould refer to the related section to see how it is managed.\#<This
is the test comment, but the test is enough long, and there is nothing more
to say.>\# 

\begin{algor}[1]
\item [{*}]action 1
\end{algor}
\item [sinon]~

\begin{algor}[1]
\item [{*}]action 2
\item [{*}]action 3 \#<comment here>\#
\item [{*}]action that is quite difficult to explain. The explaination wil take several
lines in order to show that long state items are allowed. For instance the following
formula: \( \frac{\sqrt{\cos (\delta )}}{\sin (\beta )} \) is needed to explain
some things here.\#<something to say as extra comment?>\#
\item [{*}]another action.
\end{algor}
\item [finsi]\#<a comment here could be usefull. Let's do it very long to see whether
it is well managed. At the end of this test something should be done to continue...
The problem is that \textbf{fin si} will not appear because of the noend option.>\#
\end{algor}

\subsection{Caption at the top}

The algorithm \ref{alg:top-caption} should work too.

\begin{algorithm}

\caption{\label{alg:top-caption}basic algorithm}

\begin{algor}[1]
\item [si]it is true

\begin{algor}[1]
\item [{*}]do something
\end{algor}
\item [finsi]~\end{algor}
\end{algorithm}


\end{document}
