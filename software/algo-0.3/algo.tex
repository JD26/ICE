%% LyX 1.1 created this file.  For more info, see http://www.lyx.org/.
%% Do not edit unless you really know what you are doing.
\documentclass{article}
\usepackage[T1]{fontenc}
\usepackage[latin1]{inputenc}
\usepackage{algorithm}

\makeatletter


%%%%%%%%%%%%%%%%%%%%%%%%%%%%%% LyX specific LaTeX commands.
\providecommand{\LyX}{L\kern-.1667em\lower.25em\hbox{Y}\kern-.125emX\@}

%%%%%%%%%%%%%%%%%%%%%%%%%%%%%% Textclass specific LaTeX commands.
 \newenvironment{lyxcode}
   {\begin{list}{}{
     \setlength{\rightmargin}{\leftmargin}
     \raggedright
     \setlength{\itemsep}{0pt}
     \setlength{\parsep}{0pt}
     \verbatim@font}%
    \item[]}
   {\end{list}}
 \usepackage{algolyx}
 \usepackage{algolyx}

%%%%%%%%%%%%%%%%%%%%%%%%%%%%%% User specified LaTeX commands.
% Language
% \algolang{french}

% Algorithmic possible option
% \def\algoption{noend}

% Comments delimiters redefinition
% \keycomment{\#<}{>\#}

\makeatother

\begin{document}


\title{Proposal for a \LyX{} Algorithm Style}


\author{Beno�t Guillon}

\maketitle
\tableofcontents{}

\listofalgorithms{}


\section{Introduction}

This file describes the use of a layout that allows to write algorithms without
using any \LaTeX{} command in a lyx document. The layout is built in order to
fulfill the WYSIWYM principles, and uses the \textsf{algorithm} and \textsf{algorithmic}
packages.


\section{Document class}

The \texttt{algorithm.inc} file contains the Algorithm style definition. To
be available in a document class, it must be included in the related layout.
The \texttt{article-algo.layout} is an example that supports the Algorithm style.
The \texttt{algorithm.inc} file needs the commands provided by the latex package
\texttt{algolyx.sty}, that must be installed such that \LaTeX{} can find it.
When this layout is installed (in the \texttt{\textasciitilde{}/.lyx/layout}
directory), the algolyx.sty package installed, and lyx is reconfigured, the
\texttt{Article~(algo)} class should be available. This is the current class
of this document.


\section{Version}

The package version detailed in this document is 0.3.


\section{Using the \texttt{Algorithm} style}


\subsection{Common rules}

To use the lyx algorithm style, simply select the \texttt{Algorithm} style,
and respect the following rules: 

\begin{itemize}
\item The style is derived from the Description style, and thus the first word of
each item has a special meaning. The first word you write (bold typed) must
be one of the predefined algorithmic keywords (if, else, etc.). 
\item If the first word is not a predefined keyword, then it is assimilated to the
algorithmic environment \textbackslash{}STATE keyword. In this document, the
``{*}`` character is used to define a state item.
\item For the keywords that need an extra parameter (such as: if, else~if, until,
while), the parameter is considered to be the words following the bold keyword,
until a comment starts (detected by a starting comment delimiter), or until
the end of the line.
\item If the item levels are used (only for screen viewing purpose), they must be
consistent, i.e. at the end of the algorithm, the depth level must come to zero.
\end{itemize}

\subsection{Customising the screen aspect}

The screen viewing of the algorithms you write can be customised as follow:

\begin{itemize}
\item Use the item levels to indent the algorithm blocks. Besides the screen aspect,
the item levels are not used (see section~\ref{sec:indenting}).
\item Customise the algorithm keywords with your own language keywords (see section~\ref{sec:custom}).
\item Customise the comments delimiters used (see section~\ref{sec:comments}).
\end{itemize}

\subsection{Customising the output}

The output result can be customised by:

\begin{itemize}
\item Using the \texttt{Algorithm (num)} style, to print the algorithm line numbers
(see section~\ref{sec:algo-num}).
\item Setting the global option noend, to omit the end statements in the output (see
section~\ref{sec:options}).
\item Customising the output algorithm keywords (see section~\ref{sec:custom}).
\end{itemize}

\subsection{\label{sec:indenting}Indenting the algorithm}

The screen algorithm indenting is useful to see directly on the lyx document
the algorithm block levels, without exporting it to any output format. However,
this feature is optional, and has no influence on the output aspect.

Algorithm \ref{alg:without-indent} is an example that does not use the indent
capabilities and algorithm \ref{alg:with-indent} uses the indent feature.

\begin{algorithm}
\begin{algor}
\item [if]something is true
\item [{*}]do action 1
\item [{*}]do action 2
\item [elseif](something else is true) 
\item [{*}]perform the specific action
\item [forall](items in a group)
\item [{*}]do a special action
\item [endfor]~
\item [repeat]~
\item [{*}]run action 3 
\item [until](everything is done)
\item [else]~
\item [loop]~
\item [{*}]we are sticked here
\item [endloop]~
\item [endif]~
\end{algor}

\caption{\label{alg:without-indent}Example without (screen) indentation}
\end{algorithm}


\begin{algorithm}
\begin{algor}
\item [Require:]a pre-condition must be met
\item [Ensure:]the algorithm output consistency
\item [if](something is true)

\begin{algor}
\item [{*}]do action 1
\item [{*}]do action 2
\end{algor}
\item [elseif](something else is true)

\begin{algor}
\item [{*}]perform the specific action
\item [repeat]\{comment here\}

\begin{algor}
\item [{*}]run action 3
\end{algor}
\item [until](everything is done)
\end{algor}
\item [else]~

\begin{algor}
\item [{*}]do action 4
\item [if](another thing is true)

\begin{algor}
\item [for]\( (i=0;\, i<10;\, i++) \)

\begin{algor}
\item [{*}]process the iteration
\end{algor}
\item [endfor]~
\item [while](it stills true)

\begin{algor}
\item [{*}]run action 5
\end{algor}
\item [endwhile]~
\end{algor}
\item [endif]~
\end{algor}
\item [endif]~
\end{algor}

\caption{\label{alg:with-indent}Example with (screen) indentation}
\end{algorithm}
 


\subsection{\label{sec:comments}Using comments}

Comments can be defined when enclosed in the appopriate delimiters, placed after
the bold keyword and its extra parameter. Such as for the original algorithmic
commands, comments are supported for the following keywords: if, elseif, else,
while, for, forall, repeat and loop. In addition the layout allows to put comments
after endif, endfor, endwhile, untill and endloop.

A comment can be on several lines, but in any case the closing comment delimiter
must be the last word of an item. For example, the following line causes the
layout to fail:

\begin{description}
\item [if]here is the extra parameter \{this is the comment\} unexpected words here!!
\end{description}
The following is expected:

\begin{description}
\item [if]here is the extra parameter \{the comment ends the line\}
\end{description}
By default, the comments delimiters are ``\{`` and ``\}''. These are the
delimiters used in this document. The delimiters can be redefined by using the
following latex command in the preamble:

\begin{lyxcode}
\textbackslash{}keycomment\{<begin>\}\{<end>\}
\end{lyxcode}
The delimiters can be either a single character or several characters, but cannot
be defined through macros. For instance, the following does not work:

\begin{lyxcode}
\textbackslash{}def\textbackslash{}begincom\{\textbackslash{}\#<\}

\textbackslash{}def\textbackslash{}endcom\{>\textbackslash{}\#\}

\textbackslash{}keycomment\{\textbackslash{}begincom\}\{\textbackslash{}endcom\}
\end{lyxcode}
The following works:

\begin{lyxcode}
\textbackslash{}keycomment\{\textbackslash{}\#<\}\{>\textbackslash{}\#\}
\end{lyxcode}
Algorithm \ref{alg:with-comments} is an example using comments.

\begin{algorithm}
\begin{algor}
\item [repeat]\{first comment\}

\begin{algor}
\item [if]something true \{second comment\}

\begin{algor}
\item [{*}]action 1 \{third comment\} 
\end{algor}
\item [elseif]something else is true \{fourth comment\}

\begin{algor}
\item [{*}]action 2 \{fifth comment\}
\item [for]\( (i=0;\, i<max;\, i++) \) \{iterations to do\}

\begin{algor}
\item [{*}]action 3
\end{algor}
\item [endfor]\{\( max \) is reached now\}
\end{algor}
\item [else]\{sixth comment\}

\begin{algor}
\item [{*}]last possible action
\item [loop]\{seventh comment\}

\begin{algor}
\item [{*}]let's stay here \{another comment\}
\item [while]a condition is met \{does it need a comment?\}

\begin{algor}
\item [{*}]action 3 \{this comment is very long in order to show the very-long-comment
support. Whatever long the comment is, no word should be written after the closing
comment delimiter.\} 
\end{algor}
\item [endwhile]\{now the condition is not met anymore\}
\item [forall]items in a group \{too many comments\}

\begin{algor}
\item [{*}]do something for the item
\end{algor}
\item [endfor]~
\end{algor}
\item [endloop]\{end of loop\}
\end{algor}
\item [endif]\{end of the testing block\}
\end{algor}
\item [until]it is still true \{the big loop condition\}
\end{algor}

\caption{\label{alg:with-comments}Example with comments}
\end{algorithm}



\subsection{Safety}

Because the lyx users are not supposed to know accurately the latex algorithmic
command syntax, the layout is built in order to be as safe as possible, by forgetting
the unexpected words to avoid latex errors.

The unexpected words are suppressed as follow:

\begin{itemize}
\item A bold keyword not recognised is considered to be a state item definition, and
does not appear to the output.
\item For keywords without an extra parameter that support comments (else) the characters
between the keyword and the valid comment are suppressed.
\end{itemize}
Here is an example containing unexpected characters. Check the difference between
what you see on the screen and what is produced to the output.

\begin{algor}
\item [if]it is true \{but is it?\}

\begin{algor}
\item [{*}]do the right action \{a comment here\} 
\end{algor}
\item [else]bla bla bla \{here is the real comment\} 

\begin{algor}
\item [{*}]do the other action
\end{algor}
\item [endif]bla bla too \{ending comment!\} 
\end{algor}

\subsection{\label{sec:algo-num}The \texttt{Algorithm (num)} style}

This style is derived from \texttt{Algorithm}, and has no behavioural difference.
This style allows to produce algorithms with line numbers. Algorithm \ref{alg:algo-num}
is an example using this style.

\begin{algorithm}
\begin{algor}[1]
\item [{*}]first thing to do
\item [{*}]second thing to do
\item [if]it is true \{we hope so\}

\begin{algor}[1]
\item [{*}]action 1
\end{algor}
\item [else]\{other possibility\}

\begin{algor}[1]
\item [{*}]action 2
\end{algor}
\item [endif]~
\end{algor}

\caption{\label{alg:algo-num}Algorithm with line numbers}
\end{algorithm}



\section{\label{sec:options}Options setting}

The \textsf{algorithm} package options are not supported by the layout, because
it is an internal \LyX{} stuff (algorithm floats). The single \textsf{algorithmic}
package option (\texttt{noend}) is supported by the layout. It can be set by
writing the following in the latex preamble:

\begin{lyxcode}
\textbackslash{}algoption\{noend\}
\end{lyxcode}

\section{\label{sec:custom}Customisation}

The lyx algorithm style can be used in another language by:

\begin{itemize}
\item redefining by hand in the latex preamble all the commands used by the algorithm
style, when the language translation is not directly supported or doesn't match
your need,
\item specifying the language to support, if the language translation is available.
\end{itemize}

\subsection{Full macro redefinition}


\subsubsection{Screen keywords}

It deals with the words written in the lyx document, and that appears on the
screen. In the document, the tokens that define \textbackslash{}IF, \textbackslash{}ELSE,
etc. used in the algorithms are ``if'', ``else'', etc. These are redefinable
macros. Their default definitions are:

\begin{lyxcode}
\textbackslash{}newcommand\{\textbackslash{}keyif\}\{if\}~

\textbackslash{}newcommand\{\textbackslash{}keyelseif\}\{elseif\}~

\textbackslash{}newcommand\{\textbackslash{}keyelse\}\{else\}~

\textbackslash{}newcommand\{\textbackslash{}keyendif\}\{endif\}~

\textbackslash{}newcommand\{\textbackslash{}keyfor\}\{for\}~

\textbackslash{}newcommand\{\textbackslash{}keywhile\}\{while\}~

\textbackslash{}newcommand\{\textbackslash{}keyrepeat\}\{repeat\}~

\textbackslash{}newcommand\{\textbackslash{}keyuntil\}\{until\}~

\textbackslash{}newcommand\{\textbackslash{}keyendfor\}\{endfor\}~

\textbackslash{}newcommand\{\textbackslash{}keyendwhile\}\{endwhile\}

\textbackslash{}newcommand\{\textbackslash{}keyloop\}\{loop\}~

\textbackslash{}newcommand\{\textbackslash{}keyendloop\}\{endloop\}~

\textbackslash{}newcommand\{\textbackslash{}keyrequire\}\{Require:\}~

\textbackslash{}newcommand\{\textbackslash{}keyensure\}\{Ensure:\}
\end{lyxcode}

\subsubsection{Output keywords}

To customise the algorithm words to the output, redefine the algorithmic environment
macros. The default definitions of these macros are:

\begin{lyxcode}
\textbackslash{}newcommand\{\textbackslash{}algorithmicrequire\}\{\textbackslash{}textbf\{Require:\}\}~

\textbackslash{}newcommand\{\textbackslash{}algorithmicensure\}\{\textbackslash{}textbf\{Ensure:\}\}~

\textbackslash{}newcommand\{\textbackslash{}algorithmiccomment\}{[}1{]}\{\textbackslash{}\{\#1\textbackslash{}\}\}~

\textbackslash{}newcommand\{\textbackslash{}algorithmicend\}\{\textbackslash{}textbf\{end\}\}~

\textbackslash{}newcommand\{\textbackslash{}algorithmicif\}\{\textbackslash{}textbf\{if\}\}~

\textbackslash{}newcommand\{\textbackslash{}algorithmicthen\}\{\textbackslash{}textbf\{then\}\}~

\textbackslash{}newcommand\{\textbackslash{}algorithmicelse\}\{\textbackslash{}textbf\{else\}\}~

\textbackslash{}newcommand\{\textbackslash{}algorithmicelsif\}\{\textbackslash{}algorithmicelse\textbackslash{}~\textbackslash{}algorithmicif\}~

\textbackslash{}newcommand\{\textbackslash{}algorithmicendif\}\{\textbackslash{}algorithmicend\textbackslash{}~\textbackslash{}algorithmicif\}~

\textbackslash{}newcommand\{\textbackslash{}algorithmicfor\}\{\textbackslash{}textbf\{for\}\}~

\textbackslash{}newcommand\{\textbackslash{}algorithmicforall\}\{\textbackslash{}textbf\{for~all\}\}~

\textbackslash{}newcommand\{\textbackslash{}algorithmicdo\}\{\textbackslash{}textbf\{do\}\}~

\textbackslash{}newcommand\{\textbackslash{}algorithmicendfor\}\{\textbackslash{}algorithmicend\textbackslash{}~\textbackslash{}algorithmicfor\}~

\textbackslash{}newcommand\{\textbackslash{}algorithmicwhile\}\{\textbackslash{}textbf\{while\}\}~

\textbackslash{}newcommand\{\textbackslash{}algorithmicendwhile\}\{\textbackslash{}algorithmicend\textbackslash{}~\textbackslash{}algorithmicwhile\}~

\textbackslash{}newcommand\{\textbackslash{}algorithmicloop\}\{\textbackslash{}textbf\{loop\}\}~

\textbackslash{}newcommand\{\textbackslash{}algorithmicendloop\}\{\textbackslash{}algorithmicend\textbackslash{}~\textbackslash{}algorithmicloop\}~

\textbackslash{}newcommand\{\textbackslash{}algorithmicrepeat\}\{\textbackslash{}textbf\{repeat\}\}~

\textbackslash{}newcommand\{\textbackslash{}algorithmicuntil\}\{\textbackslash{}textbf\{until\}\}
\end{lyxcode}

\subsection{Language support}

Some languages are directly supported by the algorithm style. To specify the
language to use enter the following command in the \LaTeX{} preample:

\begin{lyxcode}
\textbackslash{}algolang\{<language>\}
\end{lyxcode}
At the moment the available languages are:

\begin{itemize}
\item french
\end{itemize}

\section{Using the algorithm style in other document layouts}

To make the algorithm textclass available in another document class layout than
article-algo, you just need to do as follow:

\begin{enumerate}
\item Copy the layout to enrich in your local configuration layout directory. Example
that customises the report layout:

\begin{lyxcode}
>~cp~/usr/local/share/lyx/layouts/report.layout~\textbackslash{}~\\
~~~~~\$HOME/.lyx/layouts/report-algo.layout
\end{lyxcode}
\item Edit the copied layout, and: 

\begin{enumerate}
\item Change the the first line:

\begin{lyxcode}
\#~\textbackslash{}DeclareLaTeXClass{[}report{]}\{report~(algo)\}
\end{lyxcode}
\item Add the following lines:

\begin{lyxcode}
\#~Input~lyx~algorithm~definitions~~\\
Input~algorithm.inc
\end{lyxcode}
\end{enumerate}
\item Save the new layout, and reconfigure lyx to make the new layout available.
\end{enumerate}

\section{Changes}

Differences between the current release and the previous one (0.2):

\begin{itemize}
\item Comments after endif, endfor, endwhile, endloop, untill are now supported. 
\item The package definition is splitted in two files: \texttt{algolyx.sty} provides
all the latex commands, and \texttt{algorithm.inc} provides the new \LyX{} styles.
\item Long items (i.e. on several lines) are now supported. It includes state items,
and items using an extra parameter.
\item No words must be written after a closing comment. This new constraint seems
to be acceptable, and is a condition to have long items available.
\item The comments delimiters can be redefined (by using \texttt{\textbackslash{}keycomment}
in the preamble).
\item The global algorithmic option noend is supported (by using \texttt{\textbackslash{}algoption}
in the preamble).
\item Line numbering in algorithms is available by using the Algorithm (num) style.
\end{itemize}
Differences between version 0.2 and release 0.1:

\begin{itemize}
\item Some languages translations are supported.
\item The typewriter font previously used to display algorithms is removed: it is
not useful, and it makes algorithms too big on screen.
\end{itemize}
Differences between version 0.1 and the beta version:

\begin{itemize}
\item The commands are more robust: unexpected words are simply lost (not shown to
the output).
\item The extra parameters (for: if, for, while, etc.) do not need to be enclosed
in parenthesis anymore.
\item The require, ensure, loop, endloop forall keywords are now supported.
\item Comments are supported.
\end{itemize}

\section{Limitations -- Bugs}

This \LyX{} algorithm style has (at least) the following limitations:

\begin{itemize}
\item The line numbering interval is hard coded to 1, for the \texttt{Algorithm (num)}
style. It means that every line number is printed. There is no user option to
change this.
\item Using a list environment (itemize, description, etc.) in an algorithm does not
work.
\item The \textsf{algorithm} options are not supported.
\end{itemize}
\end{document}
